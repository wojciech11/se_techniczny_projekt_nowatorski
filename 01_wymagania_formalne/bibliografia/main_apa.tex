\documentclass[12pt,a4paper]{report}
\usepackage[utf8]{inputenc}
\usepackage{polski}
\usepackage{graphicx}

\usepackage{natbib}
\bibliographystyle{apalike}

\graphicspath{ {images/} }

\title{
{Przykład formatu cytowanie}\\
{\large Institution Name}\\
{\includegraphics[width=\textwidth]{logomerito.png}}
}
\author{Author Name}
\date{Day Month Year}

\begin{document}
\maketitle

% https://www.overleaf.com/learn/latex/Sections_and_chapters
\section*{Zauważ}

To jest przykład dla biblografii, a nie template dla pracy inżynierskiej.

\section*{Sekcja}

Tutaj cytuje artykuł (\cite{knuth:1984}) i a tutaj URL (\cite{immigrants2021a}), a tu książkę (\cite{texbook}).

Jeśli używasz MS Worda, to MS Word ma wbudowane zarządzanie bibliografią, wybierz odpowiedni style. Jako, że MS Word to MS Word, proszę regularnie backupuj swoją pracę, bo bibliografia może się rozjechać.

Nie zależnie co używasz, proszę zapisuj wszystkie artykuły czy strony internetowe na których się opierach, np., na GDrive czy na dysk.

\bibliography{main}

\end{document}
